% !TeX root = ../main.tex
\section{EXPERIMENTS}


\subsection{Dataset}
The dataset\cite{Alpher03} used in this work is the Tripitaka Koreana in Han (TKH) Dataset and the Multiple Tripitaka in Han (MTH) Dataset, which were introduced in paper [12]. To facilitate the research in Chinese historical documents, more challenging document images from the Internet are added to MTH dataset whose images number is now 2200, the combined dataset of TKH and MTH2200 is named MTHv2. Details of the dataset are given in Table I. 

In this dataset, we provide three types of annotations. The first type is line-level annotation, including text line location and its transcription, which is saved in reading order. The second type is character-level annotation, which includes class categories and bounding box coordinates. The last type is the boundary lines, represented by the start and end points of line segments. We randomly split the MTHv2 dataset into training set and testing set with the ratio of 3:1. The specific training set and test set are also be published.

\subsection{Implementation details}


\subsection{Evaluation}

