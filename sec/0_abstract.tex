\begin{abstract}
Long-tailed problem has been a long-standing research topic in research of Chinese ancient text recognition task.
However, there still exists a main challenges: the tail class usually has poor performance due to the extreme imbalance of samples.
In this paper, a novel spindle network has been proposed, which effectively alleviates the long-tail problem of ancient text recognition. 
Our method improves the accuracy of the head class by improving low-level feature extraction capabilities, thereby improving the performance of some tail classes due to the component similarities between characters.
Specifically, we reshape the number of head parameters of the feature extraction network into a spindle structure. For feature extraction networks, while the number of channels in shallow layers increases, the number of channels in deep layers decreases. In order to keep the number of parameters unchanged to the maximum extent, the total number of channels after deformation remains the same as before deformation. Compared with the mainstream model structure, the spindle network can significantly improve the feature extraction capability, thereby improving the recognition accuracy of tail category characters. Further experiments demonstrated the effectiveness of the spindle network. Extensive experiments on three challenging Chinese ancient book datasets (TKH, MTH1000 and MTH1200) verify that our method achieves the state-of-the-art performances. Channels of high level are reduced while channels of low level are increased.

% The long-tailed problem has been extensively studied in the field of Chinese ancient text recognition. However, two main challenges still persist: 1)the tail class often suffers from poor performance due to the significant imbalance in sample distribution. 2) many existing methods lack evaluation metrics specifically designed for long-tail datasets. 
% To address these challenges, this paper proposes a novel spindle network that effectively mitigates the long-tail problem in ancient text recognition. Our approach enhances the accuracy of the head class by improving the capabilities of low-level feature extraction, thereby improving the performance of certain tail classes due to the similarities between character components.
% Specifically, we introduce a spindle structure to expand the number of head parameters in the feature extraction network. Compared to mainstream model structures, the spindle network significantly enhances the feature extraction capabilities, leading to improved recognition accuracy for tail category characters. The effectiveness of the spindle network is demonstrated through extensive experiments.
% Moreover, we propose a formula for evaluating the performance of a model on Chinese ancient book datasets with long-tail distribution. This evaluation metric takes into account the specific characteristics of long-tail datasets, providing a comprehensive assessment of model performance.
% We conduct extensive experiments on three challenging Chinese ancient book datasets (TKH, MTH1000, and MTH1200) to validate the superiority of our proposed method. The results show that our approach achieves state-of-the-art performance in this field.

\end{abstract}